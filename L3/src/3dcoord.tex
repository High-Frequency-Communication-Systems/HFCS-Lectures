\documentclass{article}
\usepackage{verbatim}

% \usepackage{pgfplots} 
% \usepgfplotslibrary{external} 
% \tikzexternalize
\usepackage{tikz} 
\usepackage{physics}  
\usetikzlibrary{calc}
\usepackage{tikz-3dplot}
\usepackage[active,tightpage]{preview}  % generates a tightly fitting border around the work
\PreviewEnvironment{tikzpicture}
\setlength\PreviewBorder{2mm}

% from https://tex.stackexchange.com/a/375604/121799
%along x axis
\makeatletter
\define@key{x sphericalkeys}{radius}{\def\myradius{#1}}
\define@key{x sphericalkeys}{theta}{\def\mytheta{#1}}
\define@key{x sphericalkeys}{phi}{\def\myphi{#1}}
\tikzdeclarecoordinatesystem{x spherical}{% %%%rotation around x
    \setkeys{x sphericalkeys}{#1}%
    \pgfpointxyz{\myradius*cos(\mytheta)}{\myradius*sin(\mytheta)*cos(\myphi)}{\myradius*sin(\mytheta)*sin(\myphi)}}

%along y axis
\define@key{y sphericalkeys}{radius}{\def\myradius{#1}}
\define@key{y sphericalkeys}{theta}{\def\mytheta{#1}}
\define@key{y sphericalkeys}{phi}{\def\myphi{#1}}
\tikzdeclarecoordinatesystem{y spherical}{% %%%rotation around x
    \setkeys{y sphericalkeys}{#1}%
    \pgfpointxyz{\myradius*sin(\mytheta)*cos(\myphi)}{\myradius*cos(\mytheta)}{\myradius*sin(\mytheta)*sin(\myphi)}}


%along z axis
\define@key{z sphericalkeys}{radius}{\def\myradius{#1}}
\define@key{z sphericalkeys}{theta}{\def\mytheta{#1}}
\define@key{z sphericalkeys}{phi}{\def\myphi{#1}}
\tikzdeclarecoordinatesystem{z spherical}{% %%%rotation around x
    \setkeys{z sphericalkeys}{#1}%
    \pgfpointxyz{\myradius*sin(\mytheta)*cos(\myphi)}{\myradius*sin(\mytheta)*sin(\myphi)}{\myradius*cos(\mytheta)}}

\makeatother

% \tikzset{declare function={myx(x,y,z)=\x*sin(\y)*}}
% {{Cos[x]*Cos[y], Cos[y]*Sin[x], 
%   Sin[y]}, {-(Cos[z]*Sin[x]) - 
%    Cos[x]*Sin[y]*Sin[z], 
%   Cos[x]*Cos[z] - Sin[x]*Sin[y]*
%     Sin[z], Cos[y]*Sin[z]}, 
%  {-(Cos[x]*Cos[z]*Sin[y]) + 
%    Sin[x]*Sin[z], 
%   -(Cos[z]*Sin[x]*Sin[y]) - 
%    Cos[x]*Sin[z], Cos[y]*Cos[z]}}


\begin{document}

\tdplotsetmaincoords{60}{110}

\pgfmathsetmacro{\rvec}{.8}
\pgfmathsetmacro{\thetavec}{30}
\pgfmathsetmacro{\phivec}{55}
\pgfmathsetmacro{\dphi}{-8}
\pgfmathsetmacro{\dtheta}{8}
\pgfmathsetmacro{\drvec}{0.15}
\pgfmathsetmacro{\Rvec}{\rvec+\drvec}
\pgfmathsetmacro{\Thetavec}{\thetavec+\dtheta}
\pgfmathsetmacro{\Phivec}{\phivec+\dphi}

\pgfmathsetmacro{\h}{0}
\pgfmathsetmacro{\l}{.1}

\begin{tikzpicture}[scale=5,tdplot_main_coords]


%-----------------------
\coordinate (O) at (0,0,0);


\tdplotsetcoord{P}{\rvec}{\thetavec}{\phivec}
\tdplotsetcoord{P1}{\Rvec}{\thetavec}{\phivec}
\tdplotsetcoord{P2}{\Rvec}{\Thetavec}{\phivec}
\tdplotsetcoord{P3}{\rvec}{\Thetavec}{\phivec}
\tdplotsetcoord{p}{\Rvec}{\thetavec}{\phivec}

\tdplotsetcoord{Q}{\rvec}{\thetavec}{\Phivec}
\tdplotsetcoord{Q1}{\Rvec}{\thetavec}{\Phivec}
\tdplotsetcoord{Q2}{\Rvec}{\Thetavec}{\Phivec}
\tdplotsetcoord{Q3}{\rvec}{\Thetavec}{\Phivec}

\tdplotsetcoord{r1}{\Rvec}{\thetavec}{\phivec}
\tdplotsetcoord{r2}{\Rvec + .2}{\thetavec}{\phivec}

\tdplotsetcoord{E1}{\Rvec}{\thetavec}{\phivec}
\tdplotsetcoord{E2}{\Rvec}{\thetavec + 15}{\phivec}

%draw figure contents
%--------------------
%draw the main coordinate system axes
\draw[thick,->] (0,0,0) -- (1,0,0) node[anchor=north east]{$x$};
\draw[thick,->] (0,0,0) -- (0,1,0) node[anchor=north west]{$y$};
\draw[thick,->] (0,0,0) -- (0,0,1) node[anchor=south]{$z$};

%draw a line from origin to point (P) 
\draw[,color=red] (O) -- (P);
\draw[,color=red] (O) -- (P1);

\draw[,color=red] (O) -- (P2);
\draw[,color=red] (O) -- (P3);




\draw[dashed, color=red] (O) -- (Pxy);
\draw[dashed, color=red] (P) -- (Pxy);
%
\draw[dashed, color=red] (O) -- (P2xy);
\draw[dashed, color=red] (P2) -- (P2xy);

% %draw a line from origin to point (Q) 
\draw[,color=red] (O) -- (Q);
\draw[,color=red] (O) -- (Q2);
\draw[color=red] (O) -- (Q3);

\draw[,color=red] (P) -- (P1) --(P2) --(P3)--(P);
%draw projection on xy plane, and a connecting line
\draw[dashed, color=red] (O) -- (Qxy);
\draw[dashed, color=red] (Q) -- (Qxy);
%
\draw[dashed, color=red] (O) -- (Q2xy);
\draw[dashed, color=red] (Q2) -- (Q2xy);

\pgfmathsetmacro{\Rproj}{\Rvec*sin(\Thetavec)}

\draw[color=black] (Pxy) -- (Qxy);
\draw[color=black] (P2xy) -- (Q2xy);


\tdplotdrawarc[-latex]{(O)}{0.3}{0}{\phivec}{anchor=north}{$\phi$}


\tdplotsetthetaplanecoords{\phivec}
% \begin{scope}[tdplot_rotated_coords]
% % Uncomment these lines if you want to know where x', y' and z' point to
% \draw[-latex,blue] (-1.5,0,0) -- (1.5,0,0)  node[above right]  {$x'$};
% \draw[-latex,blue] (0,-1.5,0) -- (0,1.5,0)  node[below] {$y'$};
% \draw[-latex,blue] (0,0,-1.5) -- (0,0,1.5)  node[above left]  {$z'$};
% \end{scope}


\tdplotdrawarc[tdplot_rotated_coords]{(0,0,0)}{0.5}{0}{\thetavec}{anchor=south west}{$\theta$}


\draw[dashed,tdplot_rotated_coords,color=gray] (\rvec,0,0) arc (0:90:\rvec);
\draw[dashed,tdplot_rotated_coords,color=gray] (\Rvec,0,0) arc (0:90:\Rvec);


\draw[dashed,color=gray] (\rvec,0,0) arc (0:90:\rvec);
\draw[dashed,color=gray] (\Rvec,0,0) arc (0:90:\Rvec);




\tdplotsetthetaplanecoords{\Phivec}


\draw[dashed,tdplot_rotated_coords,color=gray] (\rvec,0,0) arc (0:90:\rvec);
\draw[dashed,tdplot_rotated_coords,color=gray] (\Rvec,0,0) arc (0:90:\Rvec);

\begin{scope}[tdplot_main_coords]
\draw[blue,fill=blue!30,opacity=0.3] plot[variable=\x,domain=\thetavec:\Thetavec] 
(z spherical cs: radius = \rvec, phi = \Phivec, theta= \x)
-- plot[variable=\x,domain=\Phivec:\phivec] 
(z spherical cs: radius = \rvec, phi = \x, theta= \Thetavec)
-- plot[variable=\x,domain=\Thetavec:\thetavec] 
(z spherical cs: radius = \rvec, phi = \phivec, theta= \x) 
-- plot[variable=\x,domain=\phivec:\Phivec] 
(z spherical cs: radius = \rvec, phi = \x, theta= \thetavec);
%
\draw[blue,fill=blue!30] plot[variable=\x,domain=\thetavec:\Thetavec] 
(z spherical cs: radius = \Rvec, phi = \Phivec, theta= \x)
-- plot[variable=\x,domain=\Phivec:\phivec] 
(z spherical cs: radius = \Rvec, phi = \x, theta= \Thetavec)
-- plot[variable=\x,domain=\Thetavec:\thetavec] 
(z spherical cs: radius = \Rvec, phi = \phivec, theta= \x) 
-- plot[variable=\x,domain=\phivec:\Phivec] 
(z spherical cs: radius = \Rvec, phi = \x, theta= \thetavec);
%
% if you want to convince yourself that this works:
% \draw[blue,fill=red!30] plot[variable=\x,domain=80:40] 
% (z spherical cs: radius = \Rvec, phi = 20, theta= \x)
% -- plot[variable=\x,domain=20:60] 
% (z spherical cs: radius = \Rvec, phi = \x, theta= 40)
% -- plot[variable=\x,domain=40:80] 
% (z spherical cs: radius = \Rvec, phi = 60, theta= \x) 
% -- plot[variable=\x,domain=60:20] 
% (z spherical cs: radius = \Rvec, phi = \x, theta= 80);
% 
\end{scope}

\draw[thick,->,color=blue] (r1) -- (r2) node[anchor=east]{$\va{r}$};
\draw[thick,->,color=blue] (Q2xy) -- (P2xy) node[anchor=north]{$\va{H}$};
\draw[thick,->,color=blue] (E1) -- (E2) node[anchor=north]{$\va{E}$};

\draw[-stealth,very thick, color=green] (0,0,\h - \l/2) --  node[anchor=north east]{} (0,0,\h + \l/2) node[midway, color = black, anchor= south east]{$I \dd{l}$};



\end{tikzpicture}

% \documentclass[12pt]{article}
\usepackage{emoji}
\usepackage{fancyhdr}

\begin{document}



\end{document}
\end{document}