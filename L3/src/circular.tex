        \documentclass{article}
        \usepackage{verbatim}

        \usepackage{tikz}   %TikZ is required for this to work.  Make sure this exists before the next line

\usepackage{pgfplots}
\pgfplotsset{compat=newest}
        \usepackage{tikz-3dplot} %requires 3dplot.sty to be in same directory, or in your LaTeX installation
        \usetikzlibrary{%
            decorations.pathreplacing,%
            decorations.pathmorphing%
        }
        \usepackage[active,tightpage]{preview}  %generates a tightly fitting border around the work
        \PreviewEnvironment{tikzpicture}
        \setlength\PreviewBorder{0mm}

\usepackage{amssymb}
\usepackage{xifthen}
        \usepackage{physics}
        \begin{document}

        %Angle Definitions
        %-----------------

        %set the plot display orientation
        %syntax: \tdplotsetdisplay{\theta_d}{\phi_d}
\tdplotsetmaincoords{65}{0}

        %define polar coordinates for some vector
        %TODO: look into using 3d spherical coordinate system
        \pgfmathsetmacro{\rvec}{2}
        \pgfmathsetmacro{\thetavec}{70}
        \pgfmathsetmacro{\phivec}{60}

        %start tikz picture, and use the tdplot_main_coords style to implement the display 
        %coordinate transformation provided by 3dplot
        \begin{tikzpicture}[scale=2,tdplot_main_coords,cube/.style={very thin,black},
                    grid/.style={very thin,gray!80},
                    axis/.style={->,blue,ultra thick},
                    rotated axis/.style={->,purple,ultra thick}]



        %set up some lengths 
        %-----------------------
        \pgfmathsetmacro{\len}{2}
        \pgfmathsetmacro{\h}{0}
        \pgfmathsetmacro{\l}{1}

        %set up some coordinates 
        %-----------------------
        \coordinate (O) at (0,0,0);
        \coordinate (S) at (0,\h,0);
        \coordinate (T) at (0,-.5,0);

        % Define P through spherical coordinates
        \tdplotsetcoord{P}{\rvec}{\thetavec - 30}{\phivec}
        %draw figure contents
        %--------------------

            %draw a grid in the x-y plane
        \foreach \x in {-1,-.9,...,1}
            \foreach \y in {-1,-.9,...,1}
        {
            \draw[grid] (\x,-1) -- (\x,1);
            \draw[grid] (-1,\y) -- (1,\y);
        }

        % \tikzset{facestyle/.style={fill=lightgray,draw=black,thin,dashed, line join=round}}
        % % face "back" 
        % \begin{scope}[canvas is yx plane at z=-\len]
        %   \path[facestyle,shade] (-.5,-\len) rectangle (0,\len);
        % \end{scope}
        % % face  "left"
        % \begin{scope}[canvas is yz plane at x=-\len]
        %   \path[facestyle,shade] (-.5,-\len) rectangle (0,\len);
        % \end{scope}
        % % face "front"
        % \begin{scope}[canvas is yx plane at z=\len]
        %   \path[facestyle, shade] (-.5,-\len) rectangle (0,\len) ;
        % \end{scope}
        % % face  "right"
        % \begin{scope}[canvas is yz plane at x=\len]
        %   \path[facestyle, shade] (-.5,-\len) rectangle (0,\len);
        % \end{scope}
        % % face "top" 
        % \begin{scope}[canvas is zx plane at y =0]
        %  \path[facestyle] (-\len,-\len) node[anchor=south east]{\textcircled{\raisebox{-0.9pt}{1}}} node[anchor=north west]{\textcircled{\raisebox{-0.9pt}{2}}} rectangle (\len,\len)  ;
        % \end{scope}

        %draw the main coordinate system axes
        \draw[thin,-stealth] (0,0,0) -- (1,0,0) node[anchor=north east]{$x$};
        \draw[thin,-stealth] (0,0,0) -- (0,1,0) node[anchor=north east]{$y$};
        \draw[thin,-stealth] (0,0,0) -- (0,0,1) node[anchor=south]{$z$};

        %draw the negative coordinate system axes
        \draw[thin,dashed] (0,0,0) -- (-1,0,0);
        \draw[thin,dashed] (0,0,0) -- (0,-1,0);


        % \draw[thick,dashed] (0,0,0) -- (0,0,-.5);

        %draw a vector from origin to point (P) 
        % \draw[-stealth,color=black] (O) -- node[midway, anchor=east]{$\va{r}$} (P) ;

        %draw a vector from origin to point (P) 
        \draw[-stealth,color=black] (O) -- node[midway, anchor=south]{$\va{r}$} (P) ;

        % %draw a vector from source to point (P) 
        \draw[-stealth,color=black] (T) -- node[anchor=south east]{$\va{r}'$} (P);

                %draw a vector from source to point (P) 
        \draw[color=black] (P) node[anchor=south]{$P$};

        % draw the Dipole
        % \draw[-stealth,very thick, color=red] (0,0,\h - \l/2) --  node[anchor=north east]{} (0,0,\h + \l/2) node[midway, color = black, anchor= south east]{$I \dd{l}$};
        %     % \draw[-stealth,very thick, color=red] (0,0,\h) -- (0,0,\h) node[anchor=south]{$h$};
        % \draw[-stealth,very thick, color=red] (0,0) circle (2 cm);
        \tdplotdrawarc[tdplot_main_coords,color=red,very thick,-stealth]{(O)}{.5}{0}{360}{}{}{anchor=east}{$I \dd{l}$};


        \end{tikzpicture}

        \end{document}