
\documentclass{standalone}
\usepackage{tikz}
\usepackage{circuitikz}
\usetikzlibrary{decorations.pathmorphing,%
decorations.pathreplacing,calc%
}
\usepackage{etoolbox}
\usepackage{physics}
\makeatletter

\pgfcircdeclarebipole{}
    {\ctikzvalof{bipoles/tline/height}}{tline}
    {\ctikzvalof{bipoles/tline/height}}
    {\ctikzvalof{bipoles/tline/width}}
{   
    %% First find distance from startpoint to endpoint
    \pgfpointdiff{\pgfpointanchor{\ctikzvalof{bipole/name}start}{center}}
                 {\pgfpointanchor{\ctikzvalof{bipole/name}end}{center}}
    \pgfmathparse{veclen(\the\pgf@x,\the\pgf@y)}
    %% The coordinate system has been changed so that the origin is at the midpoint and
    %% the line is along the x axis. So shift back by half the length of the line, and 
    %% make the cylinder of width roughly the length of the line, with a 40pt setback
    %% on each side.
    \pgftransformxshift{\pgfmathresult/2-30pt}
    \pgf@circ@res@left=\dimexpr-\pgfmathresult pt+40pt\relax
    %% Here is the original function, copied directly from the source of circuittikz, 
    %% down to next %%
    \pgf@circ@res@step=.2\pgf@circ@res@right % half x axis
    \pgfsetlinewidth{\pgfkeysvalueof{/tikz/circuitikz/bipoles/thickness}\pgfstartlinewidth}
    \pgfpathellipse{\pgfpoint{\pgf@circ@res@right-\pgf@circ@res@step}{0}}
                   {\pgfpoint{\pgf@circ@res@step}{0}}
                   {\pgfpoint{0}{-\pgf@circ@res@up}}
    \pgfpathmoveto{\pgfpoint{\pgf@circ@res@right-\pgf@circ@res@step}{\pgf@circ@res@up}}
    \pgfpathlineto{\pgfpoint{\pgf@circ@res@left+\pgf@circ@res@step}{\pgf@circ@res@up}}
    \pgfpatharc{-90}{90}{-\pgf@circ@res@step and -\pgf@circ@res@up}
    \pgfpathlineto{\pgfpoint{\pgf@circ@res@right-\pgf@circ@res@step}{\pgf@circ@res@down}}
    %% I have to fill the figure to block out the original line
    \pgfsetfillcolor{white}
    \pgfusepath{draw,fill}
    %% Redraw part of the line that gets blocked by the cylinder by mistake
    \pgfpathmoveto{\pgfpoint{\pgf@circ@res@right-2*\pgf@circ@res@step}{0pt}}
    \pgfpathlineto{\pgfpoint{\pgf@circ@res@right}{0pt}}
    \pgfusepath{draw}
}
\begin{document}
\begin{circuitikz}[american voltages]
\draw
  (-1,0) to (6,0)
  (6,0) to [R, l_=$G \Delta z$] (6,2) % Dielectric Loss
  (4,0) to [C, l_=$C \Delta z$] (4,2) % Dielectric Loss
  
  (0,2)  to [R, l=$R \Delta z$] (2,2) % resistance
  (2,2) to [L, l_=$L \Delta z$] (4,2) % inductance
  (4,2) to (8,2)
  (6,0) to (8,0);


\filldraw[fill=white, line width=1pt] (-1,0) circle(0.9mm);
\filldraw[fill=white, line width=1pt] (-1,2) circle(0.9mm);


\filldraw[fill=white, line width=1pt] (8,2) circle(0.9mm);
\filldraw[fill=white, line width=1pt] (8,0) circle(0.9mm);

\draw[thick,->,color=black] (-1,2) to (0,2);
\node (D) at (-.5,2.5) {$I(z,t)$};

\draw[thick,->,color=gray, dashed] (0,0) to (0,2);
\node (V) at (0.5,1) {$V(z,t)$};

\draw[thick,->,color=black] (6,2) to (7,2);
\node (D) at (7.5,2.5) {$I(z+ \Delta z,t)$};

\draw[thick,->,color=gray, dashed] (8,0) to (8,2);
\node (V) at (9,1) {$V(z + \Delta z,t)$};

\draw[color=gray, dashed] (-1,0) to (-1,-.5);
\node (z) at (-1,-.75) {$z$};

\draw[color=gray, dashed] (8,0) to (8,-.5);
\node (z) at (8.0,-.75) {$z+ \Delta z$};

\draw[color=gray, dashed, <->] (-1,-.5) to (8,-.5);
\node (z) at (3.5,-.75) {$\Delta z$};

\end{circuitikz}
\end{document}