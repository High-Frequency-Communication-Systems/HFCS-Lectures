\documentclass{standalone}
\usepackage{tikz}
\usepackage{circuitikz}
\usetikzlibrary{decorations.pathmorphing,%
decorations.pathreplacing,calc%
}
\usepackage{etoolbox}
\usepackage{physics}
\makeatletter

\pgfcircdeclarebipole{}
    {\ctikzvalof{bipoles/tline/height}}{tline}
    {\ctikzvalof{bipoles/tline/height}}
    {\ctikzvalof{bipoles/tline/width}}
{   
    %% First find distance from startpoint to endpoint
    \pgfpointdiff{\pgfpointanchor{\ctikzvalof{bipole/name}start}{center}}
                 {\pgfpointanchor{\ctikzvalof{bipole/name}end}{center}}
    \pgfmathparse{veclen(\the\pgf@x,\the\pgf@y)}
    %% The coordinate system has been changed so that the origin is at the midpoint and
    %% the line is along the x axis. So shift back by half the length of the line, and 
    %% make the cylinder of width roughly the length of the line, with a 40pt setback
    %% on each side.
    \pgftransformxshift{\pgfmathresult/2-30pt}
    \pgf@circ@res@left=\dimexpr-\pgfmathresult pt+40pt\relax
    %% Here is the original function, copied directly from the source of circuittikz, 
    %% down to next %%
    \pgf@circ@res@step=.2\pgf@circ@res@right % half x axis
    \pgfsetlinewidth{\pgfkeysvalueof{/tikz/circuitikz/bipoles/thickness}\pgfstartlinewidth}
    \pgfpathellipse{\pgfpoint{\pgf@circ@res@right-\pgf@circ@res@step}{0}}
                   {\pgfpoint{\pgf@circ@res@step}{0}}
                   {\pgfpoint{0}{-\pgf@circ@res@up}}
    \pgfpathmoveto{\pgfpoint{\pgf@circ@res@right-\pgf@circ@res@step}{\pgf@circ@res@up}}
    \pgfpathlineto{\pgfpoint{\pgf@circ@res@left+\pgf@circ@res@step}{\pgf@circ@res@up}}
    \pgfpatharc{-90}{90}{-\pgf@circ@res@step and -\pgf@circ@res@up}
    \pgfpathlineto{\pgfpoint{\pgf@circ@res@right-\pgf@circ@res@step}{\pgf@circ@res@down}}
    %% I have to fill the figure to block out the original line
    \pgfsetfillcolor{white}
    \pgfusepath{draw,fill}
    %% Redraw part of the line that gets blocked by the cylinder by mistake
    \pgfpathmoveto{\pgfpoint{\pgf@circ@res@right-2*\pgf@circ@res@step}{0pt}}
    \pgfpathlineto{\pgfpoint{\pgf@circ@res@right}{0pt}}
    \pgfusepath{draw}
}
\begin{document}
% \begin{tikzpicture}
% \end{tikzpicture}

\begin{circuitikz}[scale=1,
    interface/.style={postaction={draw, decorate, decoration={border,angle=45, amplitude=-3mm, segment length=2mm}}}
    ]
%% Transmission Line
% \draw[-stealth,decorate,decoration={snake,amplitude=4pt,pre length=2pt,post length=1.5pt}] (-0.5,1) -- ++(2,0);
% % \node (A) at (2,1) {$\va{E}, \va{H}$};
% \node (A) at (-1,1) {$(V, I)$};

\draw (0,2) to (10,2);
% \node (B) at (2.5,1) {$Z_{0,1},(\mu_1, \varepsilon_1)$};
\draw (0,0) to (10,0);
% \draw (5,2) to[TL] (10,2);
% \node (C) at (7.5,1) {$Z_{0,2}, (\mu_2, \varepsilon_2)$};
% \draw (5,0) to[TL] (10,0);

\draw[dashed, thick, color=gray] (7,-1) to (7,3);
\draw[dashed, thick, color=gray] (8,-1) to (8,3);

\draw[dashed, color=black] (-1,2) to (0,2);
\draw[dashed, color=black] (-1,0) to (0,0);

\draw[dashed, color=black] (10,2) to (11,2);
\draw[dashed, color=black] (10,0) to (11,0);

\draw[thick,->,color=black] (2,2) to (3,2);
\node (D) at (2.5,2.5) {$I(z,t)$};

\node (V) at (0.5,1) {$V(z,t)$};

\draw (-4,-1) rectangle (-1,3);
\node (E) at (-2.5,1) {Source};

\draw (11,-1) rectangle (14,3);
\node (E) at (12.5,1) {Load};


\filldraw[fill=white, line width=1pt] (7,2) circle(0.9mm);
\filldraw[fill=white, line width=1pt] (7,0) circle(0.9mm);

\filldraw[fill=white, line width=1pt] (8,2) circle(0.9mm);
\filldraw[fill=white, line width=1pt] (8,0) circle(0.9mm);

\node (B) at (7.5,-.5) {$\Delta z$};
\draw[thick,<->,color=black] (7,-1) to (8,-1);
\end{circuitikz}
 

\end{document} 