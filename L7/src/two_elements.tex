\documentclass{standalone}
\usepackage{verbatim}

% \usepackage{pgfplots} 
% \usepgfplotslibrary{external} 
% \tikzexternalize
\usepackage{tikz} 
\usepackage{physics}  
\usetikzlibrary{calc}
\usepackage{tikz-3dplot}
\usepackage[active,tightpage]{preview}  % generates a tightly fitting border around the work
\PreviewEnvironment{tikzpicture}
\setlength\PreviewBorder{2mm}

% from https://tex.stackexchange.com/a/375604/121799
%along x axis
\makeatletter
\define@key{x sphericalkeys}{radius}{\def\myradius{#1}}
\define@key{x sphericalkeys}{theta}{\def\mytheta{#1}}
\define@key{x sphericalkeys}{phi}{\def\myphi{#1}}
\tikzdeclarecoordinatesystem{x spherical}{% %%%rotation around x
    \setkeys{x sphericalkeys}{#1}%
    \pgfpointxyz{\myradius*cos(\mytheta)}{\myradius*sin(\mytheta)*cos(\myphi)}{\myradius*sin(\mytheta)*sin(\myphi)}}

%along y axis
\define@key{y sphericalkeys}{radius}{\def\myradius{#1}}
\define@key{y sphericalkeys}{theta}{\def\mytheta{#1}}
\define@key{y sphericalkeys}{phi}{\def\myphi{#1}}
\tikzdeclarecoordinatesystem{y spherical}{% %%%rotation around x
    \setkeys{y sphericalkeys}{#1}%
    \pgfpointxyz{\myradius*sin(\mytheta)*cos(\myphi)}{\myradius*cos(\mytheta)}{\myradius*sin(\mytheta)*sin(\myphi)}}


%along z axis
\define@key{z sphericalkeys}{radius}{\def\myradius{#1}}
\define@key{z sphericalkeys}{theta}{\def\mytheta{#1}}
\define@key{z sphericalkeys}{phi}{\def\myphi{#1}}
\tikzdeclarecoordinatesystem{z spherical}{% %%%rotation around x
    \setkeys{z sphericalkeys}{#1}%
    \pgfpointxyz{\myradius*sin(\mytheta)*cos(\myphi)}{\myradius*sin(\mytheta)*sin(\myphi)}{\myradius*cos(\mytheta)}}

\makeatother

% \tikzset{declare function={myx(x,y,z)=\x*sin(\y)*}}
% {{Cos[x]*Cos[y], Cos[y]*Sin[x], 
%   Sin[y]}, {-(Cos[z]*Sin[x]) - 
%    Cos[x]*Sin[y]*Sin[z], 
%   Cos[x]*Cos[z] - Sin[x]*Sin[y]*
%     Sin[z], Cos[y]*Sin[z]}, 
%  {-(Cos[x]*Cos[z]*Sin[y]) + 
%    Sin[x]*Sin[z], 
%   -(Cos[z]*Sin[x]*Sin[y]) - 
%    Cos[x]*Sin[z], Cos[y]*Cos[z]}}


\begin{document}

%Angle Definitions
%-----------------

%set the plot display orientation
%syntax: \tdplotsetdisplay{\theta_d}{\phi_d}
\tdplotsetmaincoords{65}{120}


%define polar coordinates for some vector
%TODO: look into using 3d spherical coordinate system
\pgfmathsetmacro{\rvec}{3}
\pgfmathsetmacro{\thetavec}{70}
\pgfmathsetmacro{\phivec}{60}

\begin{tikzpicture}[scale=2,tdplot_main_coords,cube/.style={very thin,black},
        grid/.style={very thin,gray!80},
        axis/.style={->,blue,ultra thick},
        rotated axis/.style={->,purple,ultra thick}]


    %-----------------------
    \coordinate (O) at (0,0,0);

    \pgfmathsetmacro{\len}{2}
    \pgfmathsetmacro{\h}{-0}
    \pgfmathsetmacro{\l}{1}

    %set up some coordinates 
    %-----------------------
    \coordinate (O) at (0,0,0);
    \coordinate (S) at (0,\h,0);
    \coordinate (P1) at (-.2,0,\l/2);
    \coordinate (P2) at (.2,0,\l/2);
    \coordinate (P3) at (-.2,0,-\l/2);
    \coordinate (P4) at (.2,0,-\l/2);
    \coordinate (P_1) at (0,0,\l/2);
    \coordinate (P_2) at (0,0,-\l/2);

    % Define P through spherical coordinates
    \tdplotsetcoord{P}{\rvec}{\thetavec - 30}{\phivec}
    %draw figure contents
    %--------------------

    %     %draw a grid in the x-y plane
    % \foreach \x in {-1,-.75,...,1}
    %     \foreach \y in {-1,-.75,...,1}
    % {
    %     \draw[grid] (\x,-1) -- (\x,1);
    %     \draw[grid] (-1,\y) -- (1,\y);
    % }


    %draw figure contents
    %--------------------
    %draw the main coordinate system axes
    \draw[thick,->] (0,0,0) -- (1,0,0) node[anchor=north east]{$x$};
    \draw[thick,->] (0,0,0) -- (0,1,0) node[anchor=north west]{$y$};
    \draw[thick,->] (0,0,-1) -- (0,0,1) node[anchor=south]{$z$};


    % \tdplotdrawarc[-latex]{(O)}{0.3}{0}{\phivec}{anchor=north}{$\phi$}


    \tdplotsetthetaplanecoords{\phivec}


    \tdplotdrawarc[tdplot_rotated_coords]{(0,0,0)}{0.2}{0}{\thetavec - 30}{anchor=east}{$\theta$}

    \tdplotdrawarc[tdplot_rotated_coords]{(0,0,0)}{.8}{0}{\thetavec - 50}{anchor=south west}{$\theta_1$}

    \tdplotdrawarc[tdplot_rotated_coords]{(-1,0,0)}{.8}{0}{\thetavec - 56}{anchor=north west}{$\theta_2$}


    \draw[-stealth,color=black] (O) -- node[midway, anchor= north east]{$r$} (P) ;

    \draw[-stealth,color=black] (P_1) -- node[midway, anchor= south east]{$R_1 $} (P) ;

    \draw[-stealth,color=black] (P_2) -- node[midway, anchor=west]{$R_2$} (P) ;
    % %draw a vector from source to point (P) 
    % \draw[-stealth,color=black] (S) -- node[anchor=south east]{$\bf{r'}$} (P);

    %draw a vector from source to point (P) 
    \draw[color=black] (P) node[anchor=south]{$P$};

    %         \draw[color=black] (P1) -- (P2);
    % \draw[color=black] (P3) -- (P4);

    % draw the Dipole
    \draw[thick, color=red] (0,\h - \l/2,\l/2) --  node[anchor=north east]{} (0,\h + \l/2,\l/2) node[midway,color = black, anchor= east]{$$};

    % draw the Dipole
    \draw[thick, color=red] (0,\h - \l/2,-\l/2) --  node[anchor=north east]{} (0,\h + \l/2,-\l/2) node[midway,color = black, anchor= east]{$$};

       % draw the Dipole
       \draw[thick, color=red] (0,\h - \l/2,-\l/2) --  node[anchor=north east]{} (0,\h + \l/2,-\l/2) node[midway,color = black, anchor= east]{$$};






\end{tikzpicture}

% \documentclass[12pt]{article}
\usepackage{emoji}
\usepackage{fancyhdr}

\begin{document}



\end{document}
\end{document}

%         %Angle Definitions
%         %-----------------

%         %set the plot display orientation
%         %syntax: \tdplotsetdisplay{\theta_d}{\phi_d}
%         \tdplotsetmaincoords{60}{130}

%         %define polar coordinates for some vector
%         %TODO: look into using 3d spherical coordinate system
%         \pgfmathsetmacro{\rvec}{10}
%         \pgfmathsetmacro{\thetavec}{70}
%         \pgfmathsetmacro{\phivec}{60}

%         %start tikz picture, and use the tdplot_main_coords style to implement the display 
%         %coordinate transformation provided by 3dplot
%         \begin{tikzpicture}[scale=2,tdplot_main_coords]



%         %set up some lengths 
%         %-----------------------
%         \pgfmathsetmacro{\len}{2}
%         \pgfmathsetmacro{\h}{0}
%         \pgfmathsetmacro{\l}{1}

%         %set up some coordinates 
%         %-----------------------
%         \coordinate (O) at (0,0,0);
%         \coordinate (S) at (0,\h,0);

%         % Define P through spherical coordinates
%         \tdplotsetcoord{P}{\rvec}{\thetavec}{\phivec}
%         %draw figure contents
%         %--------------------

%         % \tikzset{facestyle/.style={fill=lightgray,draw=black,thin,dashed, line join=round}}
%         % % face "back" 
%         % \begin{scope}[canvas is yx plane at z=-\len]
%         %   \path[facestyle,shade] (-.5,-\len) rectangle (0,\len);
%         % \end{scope}
%         % % face  "left"
%         % \begin{scope}[canvas is yz plane at x=-\len]
%         %   \path[facestyle,shade] (-.5,-\len) rectangle (0,\len);
%         % \end{scope}
%         % % face "front"
%         % \begin{scope}[canvas is yx plane at z=\len]
%         %   \path[facestyle, shade] (-.5,-\len) rectangle (0,\len) ;
%         % \end{scope}
%         % % face  "right"
%         % \begin{scope}[canvas is yz plane at x=\len]
%         %   \path[facestyle, shade] (-.5,-\len) rectangle (0,\len);
%         % \end{scope}
%         % % face "top" 
%         % \begin{scope}[canvas is zx plane at y =0]
%         %  \path[facestyle] (-\len,-\len) node[anchor=south east]{\textcircled{\raisebox{-0.9pt}{1}}} node[anchor=north west]{\textcircled{\raisebox{-0.9pt}{2}}} rectangle (\len,\len)  ;
%         % \end{scope}

%         %draw the main coordinate system axes
%         \draw[thin,-stealth] (0,0,0) -- (1,0,0) node[anchor=north east]{$x$};
%         \draw[thin,-stealth] (0,0,0) -- (0,1,0) node[anchor=north east]{$z$};
%         \draw[thin,-stealth] (0,0,0) -- (0,0,1) node[anchor=south]{$y$};

%         %draw the negative coordinate system axes
%         \draw[thin,dashed] (0,0,0) -- (-1,0,0);
%         \draw[thin,dashed] (0,0,0) -- (0,0,-1);
%         % \draw[thick,dashed] (0,0,0) -- (0,0,-.5);

%         %draw a vector from origin to point (P) 
%         \draw[-stealth,color=black] (O) -- node[midway, anchor=east]{$\va{r}$} (P) ;

%         % %draw a vector from source to point (P) 
%         % \draw[-stealth,color=black] (S) -- node[anchor=south east]{$\bf{r'}$} (P);

%                 %draw a vector from source to point (P) 
%         \draw[color=black] (P) node[anchor=south]{$P$};

%         % draw the Dipole
%         \draw[,->, thick, color=red] (0,\h - \l/2,0) --  node[anchor=north east]{} (0,\h + \l/2,0) node[color = black, anchor= east]{$I$};


% \tdplotdrawarc[tdplot_rotated_coords]{(0,0,0)}{0.5}{0}{\thetavec}{anchor=south west}{$\theta$}
%         % % Draw the angle 
%         % \tdplotdrawarc[tdplot_rotated_coords]{(0,0,0)}{\rvec - 5}{\thetavec}{\phivec}{anchor=south west}{$\theta$}


%         \end{tikzpicture}

%         \end{document}