% Options for packages loaded elsewhere
\PassOptionsToPackage{unicode}{hyperref}
\PassOptionsToPackage{hyphens}{url}
%
\documentclass[
]{article}
\usepackage{amsmath,amssymb}
\usepackage{lmodern}
\usepackage{ifxetex,ifluatex}
\ifnum 0\ifxetex 1\fi\ifluatex 1\fi=0 % if pdftex
  \usepackage[T1]{fontenc}
  \usepackage[utf8]{inputenc}
  \usepackage{textcomp} % provide euro and other symbols
\else % if luatex or xetex
  \usepackage{unicode-math}
  \defaultfontfeatures{Scale=MatchLowercase}
  \defaultfontfeatures[\rmfamily]{Ligatures=TeX,Scale=1}
\fi
% Use upquote if available, for straight quotes in verbatim environments
\IfFileExists{upquote.sty}{\usepackage{upquote}}{}
\IfFileExists{microtype.sty}{% use microtype if available
  \usepackage[]{microtype}
  \UseMicrotypeSet[protrusion]{basicmath} % disable protrusion for tt fonts
}{}
\makeatletter
\@ifundefined{KOMAClassName}{% if non-KOMA class
  \IfFileExists{parskip.sty}{%
    \usepackage{parskip}
  }{% else
    \setlength{\parindent}{0pt}
    \setlength{\parskip}{6pt plus 2pt minus 1pt}}
}{% if KOMA class
  \KOMAoptions{parskip=half}}
\makeatother
\usepackage{xcolor}
\IfFileExists{xurl.sty}{\usepackage{xurl}}{} % add URL line breaks if available
\IfFileExists{bookmark.sty}{\usepackage{bookmark}}{\usepackage{hyperref}}
\hypersetup{
  pdftitle={High Frequency Communication Systems},
  pdfauthor={Hasan Tahir Abbas \& Qammer Abbasi},
  hidelinks,
  pdfcreator={LaTeX via pandoc}}
\urlstyle{same} % disable monospaced font for URLs
\usepackage{color}
\usepackage{fancyvrb}
\newcommand{\VerbBar}{|}
\newcommand{\VERB}{\Verb[commandchars=\\\{\}]}
\DefineVerbatimEnvironment{Highlighting}{Verbatim}{commandchars=\\\{\}}
% Add ',fontsize=\small' for more characters per line
\newenvironment{Shaded}{}{}
\newcommand{\AlertTok}[1]{\textcolor[rgb]{1.00,0.00,0.00}{\textbf{#1}}}
\newcommand{\AnnotationTok}[1]{\textcolor[rgb]{0.38,0.63,0.69}{\textbf{\textit{#1}}}}
\newcommand{\AttributeTok}[1]{\textcolor[rgb]{0.49,0.56,0.16}{#1}}
\newcommand{\BaseNTok}[1]{\textcolor[rgb]{0.25,0.63,0.44}{#1}}
\newcommand{\BuiltInTok}[1]{#1}
\newcommand{\CharTok}[1]{\textcolor[rgb]{0.25,0.44,0.63}{#1}}
\newcommand{\CommentTok}[1]{\textcolor[rgb]{0.38,0.63,0.69}{\textit{#1}}}
\newcommand{\CommentVarTok}[1]{\textcolor[rgb]{0.38,0.63,0.69}{\textbf{\textit{#1}}}}
\newcommand{\ConstantTok}[1]{\textcolor[rgb]{0.53,0.00,0.00}{#1}}
\newcommand{\ControlFlowTok}[1]{\textcolor[rgb]{0.00,0.44,0.13}{\textbf{#1}}}
\newcommand{\DataTypeTok}[1]{\textcolor[rgb]{0.56,0.13,0.00}{#1}}
\newcommand{\DecValTok}[1]{\textcolor[rgb]{0.25,0.63,0.44}{#1}}
\newcommand{\DocumentationTok}[1]{\textcolor[rgb]{0.73,0.13,0.13}{\textit{#1}}}
\newcommand{\ErrorTok}[1]{\textcolor[rgb]{1.00,0.00,0.00}{\textbf{#1}}}
\newcommand{\ExtensionTok}[1]{#1}
\newcommand{\FloatTok}[1]{\textcolor[rgb]{0.25,0.63,0.44}{#1}}
\newcommand{\FunctionTok}[1]{\textcolor[rgb]{0.02,0.16,0.49}{#1}}
\newcommand{\ImportTok}[1]{#1}
\newcommand{\InformationTok}[1]{\textcolor[rgb]{0.38,0.63,0.69}{\textbf{\textit{#1}}}}
\newcommand{\KeywordTok}[1]{\textcolor[rgb]{0.00,0.44,0.13}{\textbf{#1}}}
\newcommand{\NormalTok}[1]{#1}
\newcommand{\OperatorTok}[1]{\textcolor[rgb]{0.40,0.40,0.40}{#1}}
\newcommand{\OtherTok}[1]{\textcolor[rgb]{0.00,0.44,0.13}{#1}}
\newcommand{\PreprocessorTok}[1]{\textcolor[rgb]{0.74,0.48,0.00}{#1}}
\newcommand{\RegionMarkerTok}[1]{#1}
\newcommand{\SpecialCharTok}[1]{\textcolor[rgb]{0.25,0.44,0.63}{#1}}
\newcommand{\SpecialStringTok}[1]{\textcolor[rgb]{0.73,0.40,0.53}{#1}}
\newcommand{\StringTok}[1]{\textcolor[rgb]{0.25,0.44,0.63}{#1}}
\newcommand{\VariableTok}[1]{\textcolor[rgb]{0.10,0.09,0.49}{#1}}
\newcommand{\VerbatimStringTok}[1]{\textcolor[rgb]{0.25,0.44,0.63}{#1}}
\newcommand{\WarningTok}[1]{\textcolor[rgb]{0.38,0.63,0.69}{\textbf{\textit{#1}}}}
\usepackage{longtable,booktabs,array}
\usepackage{calc} % for calculating minipage widths
% Correct order of tables after \paragraph or \subparagraph
\usepackage{etoolbox}
\makeatletter
\patchcmd\longtable{\par}{\if@noskipsec\mbox{}\fi\par}{}{}
\makeatother
% Allow footnotes in longtable head/foot
\IfFileExists{footnotehyper.sty}{\usepackage{footnotehyper}}{\usepackage{footnote}}
\makesavenoteenv{longtable}
\setlength{\emergencystretch}{3em} % prevent overfull lines
\providecommand{\tightlist}{%
  \setlength{\itemsep}{0pt}\setlength{\parskip}{0pt}}
\setcounter{secnumdepth}{-\maxdimen} % remove section numbering
\directlua{luaotfload.add_fallback(
             "myfallback",
             {"NotoColorEmoji:mode=harf;"}
           )}
\ifluatex
  \usepackage{selnolig}  % disable illegal ligatures
\fi

\title{High Frequency Communication Systems}
\author{Hasan T Abbas \& Qammer H Abbasi}
\date{Spring 2022}

\begin{document}
\maketitle

\hypertarget{preliminary-information}{%
\section{Preliminary Information}\label{preliminary-information}}

\hypertarget{course-introduction}{%
\subsection{Course Introduction}\label{course-introduction}}

\begin{itemize}
\tightlist
\item
  Introduction to Millimetre wave (mmWave) and Terahertz (THz) Frequency
  Communication Systems
\item
  Theory of Electromagnetic wave propagation
\item
  Channel Modelling Schemes
\item
  Antenna Analysis and Design
\end{itemize}

\hypertarget{course-objectives}{%
\subsection{Course Objectives}\label{course-objectives}}

\begin{itemize}
\tightlist
\item
  State of the art of electromagnetic simulation strategies for THz
  devices
\item
  Application of solid-state structures and novel 2D materials in mmWave
  and THz device technologies
\item
  Antenna design with emphasis on phased arrays used for beamforming
\item
  Wireless Propagation models of mmWave and THz communication channels
\end{itemize}

\hypertarget{course-intended-learning-outcomes}{%
\subsection{Course Intended Learning
Outcomes}\label{course-intended-learning-outcomes}}

By the end of this course, you will be able to: It 😺 lorem 👅 \LaTeX.

\begin{enumerate}
\def\labelenumi{\arabic{enumi}.}
\tightlist
\item
  Recognise the physical limitations of electromagnetic wave propagation
  and the need to move to higher frequencies in next generation mobile
  communication.
\item
  Analyse the wireless channel models to characterise a cellular
  communication environment.
\item
  Use electromagnetic simulation techniques to study antennas and wave
  propagation.
\item
  Design complex antenna systems with specific beamforming needs for
  mobile environments.
\end{enumerate}

\hypertarget{course-assessments}{%
\subsection{Course Assessments}\label{course-assessments}}

\begin{longtable}[]{@{}lc@{}}
\toprule
Assessment Type & Weightage \\
\midrule
\endhead
Homework & 30 \% \\
Final Exam & 20 \% \\
Lab Exercises & 20 \% \\
Lab Project and Presentation & 20 \% \\
\bottomrule
\end{longtable}

\hypertarget{electromagnetic-theory-old_key}{%
\section{Electromagnetic Theory
:old\_key:}\label{electromagnetic-theory-old_key}}

\hypertarget{electromagnetic-waves}{%
\subsection{Electromagnetic Waves}\label{electromagnetic-waves}}

\hypertarget{tikz-picture}{%
\subsection{TikZ picture}\label{tikz-picture}}

\begin{itemize}
\tightlist
\item
  Here is a TikZ picutre
\end{itemize}

\begin{tikzpicture}
\draw (0,0) circle (2cm);
\end{tikzpicture}

\begin{itemize}
\tightlist
\item
\item
  point to a location in memory
\item
  declaration: \texttt{int\ *intPtr;}
\item
  getting an address: \texttt{intPtr\ =\ \&x;}
\item
  ``dereferencing'' a pointer gets the value pointed to:
  \texttt{*intPtr}
\end{itemize}

\hypertarget{pointer-arithmetic}{%
\subsection{Pointer arithmetic}\label{pointer-arithmetic}}

\begin{itemize}
\tightlist
\item
  In general, we cannot perform arbitrary assignments to a pointer and
  expect to read valid memory (often results in segfaults)
\item
  Exception: we can add or subtract from a pointer to navigate an array
\item
  Incrementing a pointer increments by \texttt{sizeof(type)} being
  pointed to, not by 1 memory address
\end{itemize}

\hypertarget{strings}{%
\subsection{Strings}\label{strings}}

\begin{itemize}
\tightlist
\item
  \texttt{int\ strlen(char\ *string)}
\item
  \texttt{int\ strcpy(char\ *dst,\ char\ *src)}
\item
  \texttt{int\ strcmp(char\ *str1,\ char\ *str2)}
\end{itemize}

\hypertarget{strlen}{%
\subsection{strlen}\label{strlen}}

. . .

\begin{Shaded}
\begin{Highlighting}[]
\DataTypeTok{int}\NormalTok{ strlen(}\DataTypeTok{char}\NormalTok{ *string)\{}
    \DataTypeTok{int}\NormalTok{ n;}
    \ControlFlowTok{for}\NormalTok{ (n = }\DecValTok{0}\NormalTok{; *s != }\CharTok{\textquotesingle{}\textbackslash{}0\textquotesingle{}}\NormalTok{, s++)\{}
\NormalTok{        n++;}
\NormalTok{    \}}
    \ControlFlowTok{return}\NormalTok{ n;}
\NormalTok{\}}
\end{Highlighting}
\end{Shaded}

\hypertarget{strcpy}{%
\subsection{strcpy}\label{strcpy}}

. . .

\begin{Shaded}
\begin{Highlighting}[]
\DataTypeTok{char}\NormalTok{ *strcpy(}\DataTypeTok{char}\NormalTok{ *dst, }\DataTypeTok{char}\NormalTok{ *src)\{}
    \DataTypeTok{int}\NormalTok{ i = }\DecValTok{0}\NormalTok{;}
    \ControlFlowTok{while}\NormalTok{ ((dst[i] = src[i]) != }\CharTok{\textquotesingle{}\textbackslash{}0\textquotesingle{}}\NormalTok{)\{}
\NormalTok{        i++;}
\NormalTok{    \}}
    \ControlFlowTok{return}\NormalTok{ dst;}
\NormalTok{\}}
\end{Highlighting}
\end{Shaded}

. . .

\begin{itemize}
\tightlist
\item
  src and dst cannot be overlapping, why?
\end{itemize}

\hypertarget{strcpy-with-pointer-arithmetic}{%
\subsection{strcpy with pointer
arithmetic}\label{strcpy-with-pointer-arithmetic}}

\begin{Shaded}
\begin{Highlighting}[]
\DataTypeTok{char}\NormalTok{ *strcpy(}\DataTypeTok{char}\NormalTok{ *dst, }\DataTypeTok{char}\NormalTok{ *src)\{}
    \ControlFlowTok{while}\NormalTok{ (*dst++ = *src++);}
    \ControlFlowTok{return}\NormalTok{ dst;}
\NormalTok{\}}
\end{Highlighting}
\end{Shaded}

\begin{itemize}
\tightlist
\item
  How does this work?
\end{itemize}

\hypertarget{strcmp}{%
\subsection{strcmp}\label{strcmp}}

\begin{Shaded}
\begin{Highlighting}[]
\DataTypeTok{int}\NormalTok{ strcmp(}\DataTypeTok{char}\NormalTok{ *s, }\DataTypeTok{char}\NormalTok{ *t)\{}
    \ControlFlowTok{for}\NormalTok{ ( ; *s == *t; s++, t++)\{}
        \ControlFlowTok{if}\NormalTok{ (*s == }\CharTok{\textquotesingle{}\textbackslash{}0\textquotesingle{}}\NormalTok{)\{}
            \ControlFlowTok{return} \DecValTok{0}\NormalTok{;}
\NormalTok{        \}}
\NormalTok{    \}}
    \ControlFlowTok{return}\NormalTok{ *s {-} *t;}
\NormalTok{\}}
\end{Highlighting}
\end{Shaded}

\hypertarget{the-problem-with-strlen-strcpy-strcmp}{%
\subsection{The Problem with strlen, strcpy,
strcmp}\label{the-problem-with-strlen-strcpy-strcmp}}

\begin{quote}
\begin{itemize}
\tightlist
\item
  What's wrong with these functions?
\item
  C does not store length alongside arrays
\item
  Naive versions can potentially access memory improperly if given
  non-terminated strings or insufficient space
\end{itemize}
\end{quote}

\hypertarget{buffer-overflow}{%
\subsection{Buffer overflow}\label{buffer-overflow}}

\begin{Shaded}
\begin{Highlighting}[]
\DataTypeTok{char}\NormalTok{           A[}\DecValTok{8}\NormalTok{] = }\StringTok{""}\NormalTok{;}
\DataTypeTok{unsigned} \DataTypeTok{short}\NormalTok{ B    = }\DecValTok{1979}\NormalTok{;}
\NormalTok{strcpy(A, }\StringTok{"excessive"}\NormalTok{);}
\end{Highlighting}
\end{Shaded}

. . .

!

. . .

\note{Show example here}

\hypertarget{solution-passing-the-lengths-of-string-to-functions}{%
\subsection{Solution: Passing the lengths of string to
functions}\label{solution-passing-the-lengths-of-string-to-functions}}

\begin{itemize}
\tightlist
\item
  The C standard library includes string functions which take the
  maximum length of the string as arguments (eg. \texttt{strnlen},
  \texttt{strncpy}, etc.)
\item
  These versions are less prone to buffer overruns
\item
  more info: \url{http://www.cplusplus.com/reference/cstring/}
\item
  standard library functions also have man pages (eg.
  \texttt{man\ strcpy}, \texttt{man\ malloc}, etc.)
\end{itemize}

\hypertarget{memory-allocation}{%
\subsection{Memory Allocation}\label{memory-allocation}}

\note{What's the difference between}

\begin{itemize}
\tightlist
\item
  stack
\item
  heap
\item
  static
\end{itemize}

\hypertarget{example}{%
\subsection{Example}\label{example}}

\begin{Shaded}
\begin{Highlighting}[numbers=left,,]
\NormalTok{\textbackslash{}include\{}\StringTok{"src/allocation.c"}\NormalTok{\}}
\end{Highlighting}
\end{Shaded}

\hypertarget{the-stack}{%
\subsection{The Stack}\label{the-stack}}

Assume \texttt{DrawSquare()} calls \texttt{DrawLine()}

\hypertarget{lifetime-of-a-variable}{%
\subsection{Lifetime of a variable}\label{lifetime-of-a-variable}}

\begin{quote}
\begin{itemize}
\tightlist
\item
  stack: lives until the end of the function call in which it is defined
\item
  heap: lives until it is freed
\item
  static: until program ends
\end{itemize}
\end{quote}

\hypertarget{malloc}{%
\subsection{malloc}\label{malloc}}

\begin{itemize}
\tightlist
\item
  to allocate memory on the heap, use \texttt{malloc}
\item
  function signature:
\end{itemize}

\begin{Shaded}
\begin{Highlighting}[]
\DataTypeTok{void}\NormalTok{* malloc (}\DataTypeTok{size\_t}\NormalTok{ size);}
\end{Highlighting}
\end{Shaded}

\begin{itemize}
\tightlist
\item
  we \emph{must} free memory allocated by malloc
\item
  failure to do so is a memory leak
\item
  freeing a pointer while others still hold references to it is also a
  potential error
\end{itemize}

\hypertarget{malloc-details-kr-8.7}{%
\subsection{malloc details (K\&R 8.7)}\label{malloc-details-kr-8.7}}

\begin{itemize}
\tightlist
\item
  keeps a list of free blocks of memory
\item
  each block contains a size, pointer to next block, and the free space
\item
  when a request is made, the list is scanned until a block big enough
  is found
\item
  when a block is found, it is returned and removed from the free list
\item
  if no sufficiently large block exists, ask the OS for more
\end{itemize}

\hypertarget{free-details-kr-8.7}{%
\subsection{free details (K\&R 8.7)}\label{free-details-kr-8.7}}

\begin{itemize}
\tightlist
\item
  scans the free list looking for the freed block's address
\item
  adds an entry to the list if between two blocks
\item
  merges free blocks if adjacent
\end{itemize}

\hypertarget{more-tools}{%
\section{More Tools}\label{more-tools}}

\hypertarget{valgrind}{%
\subsection{Valgrind}\label{valgrind}}

\begin{itemize}
\tightlist
\item
  Valgrind is a set of debugging and profiling tools
\item
  The most common use for Valgrind is checking for memory errors
\end{itemize}

\hypertarget{example-c-program}{%
\subsection{Example C program}\label{example-c-program}}

\begin{Shaded}
\begin{Highlighting}[numbers=left,,]
\NormalTok{\textbackslash{}include\{}\StringTok{"src/memleak.c"}\NormalTok{\}}
\end{Highlighting}
\end{Shaded}

\hypertarget{errors-in-previous-program}{%
\subsection{Errors in previous
program}\label{errors-in-previous-program}}

\begin{quote}
\begin{itemize}
\tightlist
\item
  problem 1: heap block overrun
\item
  problem 2: memory leak -- x not freed
\end{itemize}
\end{quote}

\hypertarget{lets-run-valgrind-on-this-program}{%
\subsection{Let's run valgrind on this
program}\label{lets-run-valgrind-on-this-program}}

\end{document}
